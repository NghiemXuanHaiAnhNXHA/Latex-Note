\documentclass[12pt]{article}
\title{LaTeX - Tutorial for Beginners}
\usepackage[utf8]{inputenc}
%Package for math
\usepackage{amsmath, amssymb, amsfonts}
%Package for align picture and table
\usepackage{float}
%Package to setting the margin
\usepackage[top=1in, bottom=1in, left=0.75in, right=0.75in]{geometry}
%Package to insert picture
\usepackage{graphicx}
\date{}

\begin{document}
	\tableofcontents
	\maketitle
	\section{Common Mathematical Notation}
	
	\leavevmode
	
	supperscripts $$2x^3$$
	$$2x^12$$
	$$2x^{12}$$
	
	subscripts
	$$x_{12}$$
	$$x_{1_2}$$
	$$a_0, a_1, \ldots, a_{100}$$
	
	Greek letters
	$$\pi$$
	$$\Pi$$
	$$\alpha$$
	$$A = \pi r^2$$
	
	Trig functions
	$$y = \sin x$$
	$$y = \cos x$$
	$$y = \sin^{-1} \theta$$
	
	Log functions
	$$y = \log x$$
	$$y = \log_5 x$$
	$$y = \ln x$$
	
	Roots
	$$y = \sqrt{x}$$
	$$y = \sqrt[3]{2}$$
	$$f(x) = \sqrt{x^2 + y^2}$$
	$$\sqrt{1+\sqrt{x}}$$
	
	Fractions
	$$\frac{2}{4}$$\\
	
	Normal $\frac{2}{4}$\\[16pt]
	
	Display $\displaystyle \frac{2}{4}$
	
	\section{Brackets, Tables and Arrays}
	\subsection{Brackets}
	
	\leavevmode
	
	The distributive property states that $a(b+c) = ab + ac$ for all $a, b, c \in \mathbb{R}$.\\
	
	The equivalence class of $a$ is $[a]$.\\
	
	The set $A$ is defined to be $\{1, 2, 3\}$.
	
	$$\left(\frac{1}{x^2 - 1}\right)$$
	
	$$2\left \langle \frac{x}{y} \right \rangle$$
	
	$$2\left|\frac{1}{x^2 - 1} \right|$$
	
	$$ \left. \frac{dy}{dx} \right|_{x - 1}$$
	\subsection{Table}
	
	\leavevmode
	
	\begin{tabular}{|c||c|c|c|c|c|}
		\hline
		$x$ & 1 & 2 & 3 & 4 & 5 \\ \hline
		$f(x)$ & 6 & 7 & 8 & 9 & $\frac{4}{7}$ \\ \hline
	\end{tabular}
	\vspace{1cm}
	
	\begin{table}[H]
		\centering
		\def\arraystretch{1.5}
		\begin{tabular}{|c||c|c|c|c|c|}
			\hline
			$x$ & 1 & 2 & 3 & 4 & 5\\ \hline
			$f(x)$ & $\frac{1}{2}$ & 7 & 8 & 9 & $\frac{4}{7}$\\ \hline
		\end{tabular}
	\caption{These values represent the function $f(x)$.}
	\end{table}
	
	\begin{table}[H]
		\centering
		\def\arraystretch{1.5}
		\begin{tabular}{|c|c|}
			\hline
			$f(x)$ & $f'(x)$ \\ \hline
			$x>0$ & The function $f(x)$ is increasing.\\ \hline
		\end{tabular}
		\caption{The relationship between $f(x)$ and $f'(x)$.}
	\end{table}
	
	\begin{table}[H]
		\centering
		\def\arraystretch{1.5}
		\begin{tabular}{|l|p{3in}|}
			\hline
			$f(x)$ & $f'(x)$ \\ \hline
			$x>0$ & The function $f(x)$ is increasing. The function $f(x)$ is increasing. The function $f(x)$ is increasing.\\ \hline
		\end{tabular}
		\caption{The relationship between $f(x)$ and $f'(x)$.}
	\end{table}
	
	\subsection{Equation Arrays}
	
	\leavevmode
	
	\begin{align}
		5x^2 \text{text}\\
		5x^2-9=x+3\\
		\sqrt{x + \sqrt{x}} + 1 = \sqrt{3}
	\end{align}
	
	\begin{align*}
		5x^2 \text{text}\\
		5x^2-9&=x+3\\
		\sqrt{x + \sqrt{x}} + 1 &= \sqrt{3}
	\end{align*}
	
	\section{Creating Lists}
	
	\leavevmode
	
	\begin{enumerate}
		\item pencil
		\item calculator
		\item ruler
		\item notebook
			\begin{enumerate}
				\item notes
				\item homework
				\item assessments
					\begin{enumerate}
						\item tests
						\item quizzes
						\item journal entries
					\end{enumerate}
			\end{enumerate}
		\item highlighters
	\end{enumerate}
	
	\pagebreak
	
	\begin{itemize}
		\item pencil
		\item calculator
		\item ruler
		\item notebook
	\end{itemize}
	
	\section{Text Document Formatting}
	
	\leavevmode
	
	This will produce \textit{italicized} text.\\
	
	This will produce \textbf{bold face} text.\\
	
	This will produce \texttt{typewriter font} text.\\
	
	Please visit Michelle Krummel's website at: \texttt{http://michellekrummel.com}.
	
	\vspace{1cm}
	
	Please excuse my dear aunt Sally.\\
	
	Please excuse my \begin{large}
		dear aunt Sally.
	\end{large}\\
	
	Please excuse my \begin{Large}
		dear aunt Sally.
	\end{Large}\\
	
	Please excuse my \begin{huge}
		dear aunt Sally.
	\end{huge}\\
	
	Please excuse my \begin{Huge}
		dear aunt Sally.
	\end{Huge}\\
	
	Please excuse my \begin{normalsize}
		dear aunt Sally.
	\end{normalsize}\\
	
	Please excuse my \begin{small}
		dear aunt Sally.
	\end{small}\\
	
	Please excuse my \begin{scriptsize}
		dear aunt Sally.
	\end{scriptsize}\\
	
	Please excuse my \begin{tiny}
		dear aunt Sally.
	\end{tiny}
	
	\vspace{1cm}
	
	\begin{center}
		This line is centered.
	\end{center}

	\begin{flushright}
		This line is flushright.
	\end{flushright}
	
	\section{Packages, Macros, and Graphics}
	
	\leavevmode
	
	Using geometry package to control margin and paper.\\
	
	Using amsfonts, amsmath, amssymb to write special math notation.\\
	
	Using graphicx to insert picture from your device.\\
	
	Using float to align, and control the insert position of table or picture.\\
	
	\section{Calculus Notation}
	
	\leavevmode
	
	The function $f(x) = (x-3)^2+\frac{1}{2}$ has domain $\text{D}_f:(-\infty, \infty)$ and range $\text{R}_f:\left[\frac{1}{2}, \infty\right)$\\
	
	$\lim\limits_{x \to 0} f(x)$\\
	
	$\displaystyle{\lim \limits_{x \to a} \frac{f(x) - f(a)}{x- a} = f'(a)}$\\
	
	$\displaystyle{\int} \sin x \, \text{d}x = -\cos x + C$\\
	
	$\displaystyle{\int_a^b}$\\
	
	$\displaystyle{\int \limits_a^b}$\\
	
	$\displaystyle{\int \limits_{2a}^b} x^2 \, \text{d}x = \left[\frac{x^3}{3}\right]_a^b$\\
	
	$\displaystyle{\sum \limits_{n=1}^{\infty}}$\\
	
	$\displaystyle{\int_a^b f(x) \, \text{d}x = \lim \limits_{x \to \infty} \sum \limits_{k=1}^{n} f(x_k). \Delta x}$\\
	
	$\vec{v} = v_1 \vec{i} + v_2 \vec{j} = (v_1, v_2)$
	
\end{document}
